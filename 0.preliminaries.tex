\chapter{Preliminaries}
\section{What is \LaTeX?}
Simply put, \href{https://www.latex-project.org/}{\LaTeX} is a typesetting language designed for mathematical/scientific texts, but its versatility allows it to be used in a variety of scenarios. At its core, \LaTeX\ is used to create documents (articles, reports, books, journals, etc.) for academic purposes. As a language, \LaTeX\ interprets plaintext and uses \TeX\ to generate an output (usually a \texttt{.pdf} file).

\section{Why learn and use \LaTeX?}
In scientific and academic settings, \LaTeX is used extensively to make professional documents. But even for less formal and more personal purposes, \LaTeX\ is extremely useful. It is much more powerful and produces much more visually appealing documents than typical editors like Google Docs or Microsoft Word. Also, it gives you, the user, a lot of control over how you would like your writing to look. Finally, there are tons and tons of built in and/or freely available \textit{packages} which make a lot of document features like bibliographies, tables, graphics, formatting, and beyond very easy and customizable.

\section{What is Asymptote?}
\href{https://asymptote.sourceforge.io/}{Asymptote} is a language (generally) used in \LaTeX\ to generate vector graphics. Asymptote is written in its own language, but its syntax is similar to programming languages like \ttt{C++}. It is a powerful way to create a variety of images, from charts to geometry diagrams. Specifically, Asymptote is an environment in \LaTeX used to add vector graphics to documents. 

\section{Why Asymptote?}
It's true that \LaTeX\ has other alternatives to Asymptote, most notably TikZ. As far as I know, there is no reason to lean towards either of the two. However, in my experiences I have always used Asymptote, and I have come across something it wasn't capable of doing. Furthermore, I think modules like \texttt{geometry.asy} and \texttt{olympiad.asy} make Asymptote the clear winner when it comes to geometry diagrams.

\section{How do I use \LaTeX?}
I think online \LaTeX\ compilers/editors (most notably \href{http://www.overleaf.com}{Overleaf}) are really useful and user-friendly. I used Overleaf for a while (and still do for certain things), and it has great functionality. In particular, it makes it really easy to collaborate on relatively short-term and small-scale projects. 

That being said, from my experiences I can say that installing \LaTeX\ on your computer is ultimately the way to go. (For example, do people usually code in repl.it or do people use editors like VSCode?) From what I know, all \LaTeX\ softwares already include pretty much any package you could need in your projects, and they also make it easier to add your own packages and customization in the form of \ttt{.cls} and \ttt{.sty} files (more on these later). Also, compilation is much faster on your own device than it is through some other site. As an added bonus, \LaTeX\ integrates well with \ttt{git}, just as programming languages do.

I have a Mac, so I use the \href{https://www.tug.org/mactex/}{Mac\TeX} software, which includes \TeX Shop, a \LaTeX\ editor. I'm not familiar with its equivalent on Windows or other operating systems, but the internet has plenty of resources.

\hrulefill

Now that we have addressed formalities, it's time to dive into the actual learning!

